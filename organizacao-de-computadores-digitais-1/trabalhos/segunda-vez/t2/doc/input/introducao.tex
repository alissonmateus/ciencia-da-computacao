% inkscape -D -z --file=./input/smooth.svg --export-pdf=./input/smooth.pdf
\section{Introdução \label{sec:introducao}}

Este segundo trabalho da disciplina de Organização de Computadores Digitais I tem como objetivo analisar o desempenho de uma hierarquia de memórias \textit{cache} com relação à \textbf{taxa de acerto} (\textit{hit rate}). Os resultados para a análise foram gerados por meio de simulações em \textit{software}. Para isso, foi escolhido o simulador Amnesia \cite{bib:amnesia} e nele foram executados diferentes \textit{traces} sobre uma variedade de arquiteturas de memórias \textit{cache} definidas de acordo com as limitações do simulador.

O desenvolvimento deste trabalho está organizado da seguinte maneira: primeiramente é apresentado o simulador Amnesia, em seguida são comentados e mostrados os arquivos de \textit{trace} e as arquiteturas definidas pelo grupo do trabalho e então, por meio de tabelas e gráficos, são analisadas as taxas de acerto com relação as características das arquiteturas (tamanho da \textit{cache}, tamanho do bloco, associatividade, algoritmo de substituição e quantidade de \textit{caches}).

Os arquivos utilizados neste trabalho devem acompanhar este documento.


