\chapter{Bitcoin e Criptomoedas}
\label{ch:bitcoin-intro}

\section{Considerações Iniciais}

Atualmente, as principais fontes de informação sobre Bitcoin e criptomoedas são obtidas através de sites oficiais e de pesquisadores intimamente relacionados com o tema. Este capítulo apresenta o material bibliográfico e uma visão geral do tema, discorrendo sobre sua origem, desenvolvimento e valor.

\section{Origem do Material Bibliográfico}

 Todo material consultado para este trabalho encontra-se em formato digital e, exceto os documentários, disponível gratuitamente na Internet. Os sites oficiais do projeto Bitcoin são: \textit{bitcoin.org}, \textit{bitcoincore.org} e \textit{bitcointalk.org}. As principais listas de email são: \textit{bitcoin-dev}\footnote{Disponível em: \url{https://lists.linuxfoundation.org/mailman/listinfo/bitcoin-dev}. Acesso em: 12 abr. 2016.} e \textit{bitcoin-discuss}\footnote{Disponível em: \url{https://lists.linuxfoundation.org/mailman/listinfo/bitcoin-discuss}. Acesso em: 12 abr. 2016.}. Esses meios reúnem boa documentação sobre o assunto.

Materiais didáticos também estão disponíveis. A Universidade de Princeton conta com um curso online de vídeo-aulas no Coursera\footnote{Bitcoin and Cryptocurrency Technologies. Disponível em: \url{https://www.coursera.org/course/bitcointech}. Acesso em: 12 abr. 2016.} e no YouTube\footnote{Bitcoin and Cryptocurrency Technologies Online Course. Disponível em: \url{https://www.youtube.com/channel/UCNcSSleedtfyDuhBvOQzFzQ}. Acesso em: 12 abr. 2016.}, além de um livro-texto em desenvolvimento \cite{bib:princeton-book}. A Universidade de Standford e Universidade Federal de Pernambuco (UFPE) contam com disciplinas optativas sobre criptomoeda nos cursos de computação\footnote{CS 251(p): Bitcoin and Crypto Currencies. Disponível em: \url{https://crypto.stanford.edu/cs251}. Acesso em: 12 abr. 2016.}\footnote{Centro de Informática da Universidade Federal de Pernambuco (UFPE). Seminários: ``Bitcoin e as Tecnologias de Criptomoeda''.}.

Com foco nos programadores, o livro ``Mastering Bitcoin'' de Andreas Antonopoulos, que foi escrito em modo \textit{open-source} no GitHub, é uma referência --- recomendada inclusive pelo então cientista-chefe do grupo de \textit{core developers}, Gavin Andresen \cite{bib:bitcoinbook}. Andreas palestrou no Brasil em abril de 2016 no 1º coinBR Bitcoin Summit, São Paulo\footnote{Disponível em: \url{https://www.youtube.com/watch?v=ieP8kxaklUk}. Acesso em: 20 abr. 2016.}.

Na área de economia, o livro ``Bitcoin -- a moeda na era digital'' de Fernando Ulrich é uma referência recente em português \cite{bib:fernando-ulrich} juntamente com seus artigos online publicados na InfoMoney\footnote{Disponível em: \url{http://www.infomoney.com.br/blogs/moeda-na-era-digital}. Acesso em: 12 abr. 2016.} e no Instituto Ludwig von Mises -- Brasil\footnote{Disponível em: \url{http://www.mises.org.br/SearchByAuthor.aspx?id=207}. Acesso em: 12 abr. 2016.}.

Como documentários que tratam do tema, tem-se: ``Bitcoin: The End of Money As We Know It'' (2015) que desmistifica o funcionamento dos bancos centrais, conta a história do dinheiro e contextualiza o impacto do Bitcoin nesse cenário; ``The Rise and Rise of Bitcoin'' (2014) que documenta a história do Bitcoin, das principais e pioneiras startups e apresenta uma visão geral dos eventos ocorridos desde o seu surgimento; e ``Deep Web'' (2015) que, apesar do nome, aborda o caso \textit{Silk Road}: um site de comércio eletrônico de produtos ilícitos na \textit{darknet} que usava Bitcoin como moeda.

Empresas que trabalham com criptomoedas criam canais de comunicação e divulgação de material, como por exemplo a FoxBit Exchange no Brasil\footnote{Canal da FOXBIT no YouTube. Disponível em: \url{http://www.youtube.com/FoxbitBrasil}. Acesso em: 12 abr. 2016.}. Diversas palestras, vídeo-conferências e encontros de tecnologia são gravadas e também podem ser assistidas gratuitamente em sites de vídeo como o \textit{youtube.com}.

A pesquisa sobre criptomoeda é recente e vem crescendo rapidamente em qualidade e quantidade de material publicado. Tanto inovador quanto a tecnologia, é também o material bibliográfico que, em sua maioria, é publicado em modo digital e sob licenças do tipo \textit{open-source} que facilitam o acesso à informação.

\section{O que é Bitcoin?}

O surgimento do Bitcoin ocorreu em 2008 pelo anômino cientista Satoshi Nakamoto. Em um artigo publicado na Internet, Nakamoto propôs uma moeda e sistema de pagamento online, resistente ao problema do gasto duplo (\textit{double spending}), pseudoanônimo e sem necessidade de um terceiro intermediário \cite{bib:satoshi}.

Desde então, Bitcoin e demais criptomoedas estão levantando dúvidas e especulações sobre o futuro do dinheiro, dos métodos de pagamento, da regulamentação e consequentemente da política econômica \cite{bib:fmi}. Elas podem causar grande impacto no cenário da economia global, pois tem capacidade de transferir valor digitalmente e sem fronteiras pela Internet, com baixíssimas taxas e sem burocracia, além de contrariar os modelos atuais de economia keynesiana/marxiana\footnote{``[...] pode-se dizer que o Bitcoin é o arranjo monetário que mais se aproxima daquele idealizado pelos economistas da Escola Austríaca.'' \cite[p. 66]{bib:fernando-ulrich}.}\footnote{``Não é por menos que a criação de Nakamoto antagoniza tanto muitos economistas, pois se trata de muito mais do que apenas uma teoria, ou de algum modelo econométrico sem uso prático; o Bitcoin é a prova cabal de que uma moeda privada pode surgir do mercado, por meio da livre escolha dos indivíduos, sem a mínima necessidade de um decreto governamental --- algo que contraria as teorias monetárias dominantes na academia.'' (ULRICH, Fernando. Por que Satoshi Nakamoto merece o Prêmio Nobel de Economia. 15 nov. 2015. Disponível em: \url{http://www.mises.org.br/Article.aspx?id=2223}. Acesso em: 13 abr. 2016).}. Por sua característica inovadora, o Bitcoin foi considerado por alguns como a maior inovação tecnológica desde a criação da Internet e tem recebido interesse de grandes investidores \cite{bib:atencao-investidores}.

Devido a nossa falha educação sobre economia e tecnologia, entender o que é Bitcoin não é uma tarefa fácil e exige uma reeducação na área. Bitcoin é uma moeda digital e um sistema de pagamento online, peer-to-peer, de código-fonte aberto e totalmente descentralizada, isto é, não depende de uma autoridade central para emiti-la e nem para realizar pagamentos. Seu alcance é tanto quanto a Internet for possível de prover. Para enviar e receber bitcoins é necessário apenas possuir um dispositivo eletrônico conectado à Internet e capaz de executar um aplicativo de carteira. A transferência é feita diretamente de carteira para carteira, de modo pseudoanônimo e não há necessidade de criação de contas: cada usuário gera diversos endereços \textit{hash} para usar nas transações.

Outra inovação é que as transações são irreversíveis, isto é, uma vez que $A$ transferiu uma quantia $x$ para $B$ e esta transação foi aceita na blockchain (espécie de livro-razão público com o histórico de todas as transações), não é possível revertê-la. Essa característica é inerente a concepção do Bitcoin e é fundamental para evitar o problema do gasto duplo.

No momento em que este trabalho é escrito, a quantidade de unidades de bitcoin disponíveis na rede é cerca de 15,4 milhões e por definição essa quantidade cresce até atingir um total de aproximadamente 21 milhões --- sendo o crescimento atual de em média 25 unidades a cada 10 minutos e essa quantidade diminui pela metade a cada 210 mil blocos (aproximadamente a cada 4 anos) o que torna a oferta monetária previsível.

A primeira taxa de câmbio entre Bitcoin e uma moeda fiduciária ocorreu em outubro de 2009 em que 1 BTC\footnote{Código da moeda bitcoin.} valia menos que um centavo de dólar americano \cite{bib:historyofbitcoin}. Na atual cotação, em comparação com o real, 1 BTC vale R\$ 1.645,00 \footnote{Último preço na FOXBIT segundo o site Exchange War. Disponível em: \url{http://exchangewar.info/coinprice?BTC_BRL}. Acesso em: 17 abr. 2016.} (valor ainda muito baixo, pois 1 bitcoin é divisível em até 8 casas decimais e não apenas em duas como o real). Apesar de crescente, o uso de bitcoins como meio de pagamento no comércio diário ainda é pequeno devido à baixa quantidade de transações por segundo e ao pouco conhecimento da tecnologia pelos usuários. Então suas principais aplicações estão no uso como poupança e como intercâmbio em transferências de dinheiro entre pessoas de países diferentes. Gráficos e estatísticas sobre o estado da rede Bitcoin podem ser consultados no site \textit{blockchain.info}.

Entretanto, Bitcoin ainda é um sistema experimental. No momento, sua escalabilidade é seu maior problema e se solucionado causará uma secessão no dinheiro como o conhecemos hoje, podendo levar governos e bancos tradicionais à obsolescência.

\section{Desenvolvimento}

O repositório oficial do código-fonte encontrada-se no GitHub\footnote{Disponível em: \url{https://github.com/bitcoin/bitcoin}. Acesso em: 12 abr. 2016.} sob licença MIT e é coordenado pela equipe de \textit{core developers}. Cada nova funcionalidade é proposta e intensamente discutida por meio de BIP (\textit{Bitcoin Improvement Proposal}). Nakamoto participou ativamente do desenvolvimento até dezembro de 2010 e então deixou o projeto\footnote{Mais informações em: \url{https://www.youtube.com/watch?v=1VYs_zZsorU\#t=1h19m25s}. Acesso em: 12 abr. 2016.}.

O desenvolvimento do Bitcoin é frequentemente composto por debates bem rumorosos. Em janeiro de 2016, Mike Hearn, com mais de 5 anos como desenvolvedor no projeto, escreveu declarando o Bitcoin como falido e apontado os motivos \cite{bib:mike-hearn}. Por causa desse ambiente fervoroso de desenvolvimento aberto, o Bitcoin já foi declarado como falido várias outras vezes nos últimos anos\footnote{Bitcoin Obituaries lista todas as vezes em que o Bitcoin foi declarado como falido. Disponível em: \url{https://99bitcoins.com/bitcoinobituaries}. Acesso em: 12 abr. 2016.}. Mas como disse Andreas Antonopoulos em uma vídeo-entrevista \cite{bib:entrevista-andreas} (tradução livre):
\begin{quote}
``Eu acho que é importante reconhecermos o motivo de haver um debate e o que isso significa em um projeto de código aberto, em um sistema de moeda aberto e distribuído como esse. A verdade é que as pessoas não estão acostumadas a esse tipo de debate aberto. E não estão acostumadas porque a maioria das decisões em outros sistemas financeiros são tomadas a portas fechadas por um número pequeno de pessoas, que depois anunciam suas decisões sem nenhum debate. E você escuta esse anúncio autoritário (fiduciário se você preferir) que vem de cima, muito limpo, polido e escrito por publicitários. [...] Com Bitcoin, a roupa suja se lava em público [...] Como o sistema não pode ser modificado com controle autoritário e por exigir que todos concordem para que seja modificado, esses debates podem durar um tempo até atingir um consenso. E eles acontecem de uma forma pública e aberta. [...] Se você quer algo limpo, estéril, antisséptico, você elege um ditador.''
\end{quote}

À parte do software oficial, vários aplicativos de carteira podem ser encontrados para mobile, desktop, hardware e Web\footnote{Disponível em \url{https://bitcoin.org/en/choose-your-wallet}. Acesso em: 13 abr. 2016.} com diferentes níveis de segurança e funcionalidades. E com relação aos investimentos, existe um considerável interesse de empresas em startups com projetos envolvendo Bitcoin/blockchain, somando cerca de US\$ 1 bilhão até novembro de 2015 \cite{bib:investimentos}.

\section{Altcoins}

Após o surgimento do Bitcoin, diversas criptomoedas alternativas, as ``altcoins'', foram surgindo como modificações, \textit{forks}\footnote{Visualização dos \textit{forks} em: \url{http://mapofcoins.com/bitcoin}. Acesso em: 12 abr. 2016.}\footnote{Estatísticas comparando Bitcoin e demais altcoins em: \url{https://bitinfocharts.com}. Acesso em: 14 abr. 2016.}, do código-fonte original de Nakamoto. Entre os principais motivos para a criação de uma altcoin, tem-se:
\begin{itemize}
	\item \textbf{Concorrência}: altcoin que tem o mesmo propósito do Bitcoin (servir como moeda e meio de pagamento) e seu objetivo é competir tentando ser uma moeda melhor. Para isso, possui algoritmos/parâmetros e/ou protocolos diferentes e pode implementar novas funcionalidades que o Bitcoin não possui. Exemplos: Litecoin, Decred, Dash.

	\item \textbf{Inovação}: altcoin que busca um novo propósito a ser explorado com a tecnologia do Bitcoin. Exemplos: Namecoin (nomes de domínio \textit{.bit}), Ethereum (\textit{smart contracts}).

	\item \textbf{Entretenimento, didática}: altcoin cujo propósito é servir de porta de entrada para usuários que queiram ter seu primeiro contato com uma criptomoeda, mas ainda têm receio de se envolver com a tecnologia. Exemplos: Dogecoin, Dilmacoin.

	\item \textbf{Golpe (\textit{scam})}: altcoin criada com o propósito de enganar pessoas, convencendo-as a investir em uma moeda intencionalmente insegura e obscura quanto a sua oferta monetária. Seus criadores acumulam grandes quantidades da moeda e lucram vendendo-as momentos antes de seu declínio. Exemplo: Auroracoin.
\end{itemize}

\section{Mas Bitcoin Tem Valor?}

Segundo a teoria da Escola Austríaca, não existe valor intrínseco, mas sim propriedades intrínsecas (químicas, físicas e matemáticas). O valor é sempre subjetivo e está na mente/necessidade do indivíduo. No caso das criptomoedas, elas dependem de suas propriedades matemáticas e tecnológicas que propiciam a confiança dos usuários no sistema e fazem com que estes venham a valorizá-las --- o que é demonstrado quando eles livre e voluntariamente efetuam transações utilizando as criptomoedas. Logo, o valor de uma moeda depende somente das pessoas e não do material\footnote{Notoriamente o ouro é o material físico com maior valor como moeda devido as suas ótimas propriedades intrínsecas, mas vale ressaltar: seu valor é totalmente subjetivo.} em si ou de um decreto governamental\footnote{``O dinheiro não é invenção do Estado, nem resultado de um ato legislativo; portanto, sua sanção por parte da autoridade estatal é totalmente alheia ao conceito de dinheiro. Também a adoção de determinadas mercadorias como dinheiro teve sua origem em um processo natural a partir das condições econômicas existentes, sem que houvesse necessidade da interferência do Estado nesse processo.'' (MENGER, Carl. Princípios de Economia Política).}. Além disso, as moedas, assim como qualquer outra mercadoria, também estão suscetíveis à evolução, ao aprimoramento e, o mais importante, \textbf{à concorrência}.

Citando Fernando Ulrich\footnote{\cite[p. 88-89, 91]{bib:fernando-ulrich}.}:
\begin{quote}
``Moeda, então, é mais bem entendida como uma qualidade de uma mercadoria de servir como um meio de troca, como um bem que é intercambiado no mercado e circula de mão em mão sem jamais, ou por um longo período, ser consumido de fato. Tal qualidade é potencializada ou debilitada por atributos variados intrínsecos a uma mercadoria --- escassez, durabilidade, homogeneidade espacial e temporal, divisibilidade, maleabilidade, transportabilidade, etc. --- e atributos ``artificiais'' conferidos por influências externas e estrangeiras à natureza da mercadoria --- leis estatais de curso forçado, restrições legais de uso, etc. [...] moeda é qualquer bem econômico empregado indefinidamente como meio de troca, independentemente de sua liquidez frente a outros bens monetários e de seus possíveis usos alternativos.

[...]

Bitcoin é, portanto, uma moeda, um bem econômico empregado indefinidamente como meio de troca, embora com liquidez inferior à da maior parte das moedas fiduciárias nacionais neste instante da história.''
\end{quote}

E ainda, esclarece\footnote{\cite[p. 75]{bib:fernando-ulrich}.}:
\begin{quote}
``Qual o lastro do ouro? A escassez inerente a suas propriedades físico-químicas. Qual o lastro do papel-moeda fiduciário? A confiança de que governos não inflacionarão a moeda, apoiada em leis de curso forçado que obrigam os cidadãos a aceitar a moeda como pagamento. Qual o lastro do Bitcoin? Propriedades matemáticas que garantem uma oferta monetária, cujo aumento ocorre a um ritmo decrescente a um limite máximo e pré-sabido por todos os usuários da moeda. Após um bem ser empregado e reconhecido como moeda, seu lastro jaz na sua escassez relativa.

Mas qual a distinção-chave entre o lastro do ouro e o do Bitcoin e o lastro das moedas estatais? O lastro físico é naturalmente provido de ou pretende assegurar uma escassez de oferta, assim como o lastro matemático do Bitcoin. O lastro governamental, porém, garante unicamente uma demanda mínima, mas não uma oferta inelástica. Em outras palavras, o lastro estatal não assegura uma moeda boa, apenas que até uma moeda ruim tenha vasta aceitação no mercado.''
\end{quote}

\section{Considerações Finais}

Criptomoeda é uma tecnologia com grande potencial disruptivo cujas tecnologias que pretende derrubar são o próprio governo e os bancos centrais. Foi apresentada aqui uma visão geral sobre assunto, mas é ainda preciso introduzir os componentes básicos de uma criptomoeda.