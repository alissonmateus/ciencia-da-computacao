% Inicializa o documento
% define papel, tamanho global de fonte, tipo de documento
\documentclass[a4paper, twoside, 12pt]{article}

	% Pacotes usados
	\usepackage[utf8]{inputenc} % enconding de caracteres
	\usepackage[brazil]{babel}  % locale pt_BR
	\usepackage[lmargin=2cm, rmargin=2cm, tmargin=2cm, bmargin=2cm]{geometry} % margens da folha
	\usepackage{indentfirst} % sempre indenta o primeiro parágrafo

	% Matemática	
	\usepackage{amsmath}
	\usepackage{amsfonts}
	\usepackage{gensymb}
	
	% Para links e url
	\usepackage[hidelinks]{hyperref}
	
	\usepackage{listings} % listagem de código-fonte
	\renewcommand*{\lstlistingname}{Listagem} % texto para listagem de código
	\usepackage{color} % cor para usar na listagem de código-fonte
	\usepackage{svg}
	\usepackage{graphicx,xcolor} % para inserir imagens
	\usepackage[nottoc,notlot,notlof]{tocbibind} % adiciona o tópico Referências ao Sumário
	\usepackage{textcomp} % accesso \textquotesingle

	% Para desenhar grafos
	\usepackage{tikz}
	\usetikzlibrary{arrows,positioning,shapes,decorations}

	% Desenhar circulos
	\usepackage{tkz-euclide}
	\usetkzobj{all}

	% Escrever algoritmos em pseudo-código
	\usepackage[portuguese,linesnumbered]{algorithm2e}

	% Tabelas
	\usepackage{booktabs}
	\usepackage{caption}

	% Estilos para usar nos grafos
	\tikzset{
		>=stealth',
		punkt/.style={
			rectangle,
			text centered,
			inner sep=0.7em,
			draw,
			fill=blue!5
		},
		pil/.style={
			->,
			thick,
			shorten <=2pt,
			shorten >=2pt
		}
	}

	% Para plotar gráficos
	\usepackage{pgfplots}
	\pgfplotsset{width=10cm,compat=1.9}

	% Define cores para o highlight de código-fonte
	\definecolor{dkgreen}{rgb}{0,0.6,0}
	\definecolor{gray}{rgb}{0.5,0.5,0.5}
	\definecolor{mauve}{rgb}{0.58,0,0.82}
	
	% Define configuração para listagem de código-fonte em linguagem C
	\lstset{
		frame=tb,
		language=C,
		aboveskip=2mm,
		belowskip=2mm,
		showstringspaces=false,
		columns=flexible,
		basicstyle={\small\ttfamily},
		numbers=none,
		keywordstyle=\color{blue},
		commentstyle=\color{dkgreen},
		stringstyle=\color{mauve},
		breaklines=true,
		breakatwhitespace=false,
		tabsize=4
	}

% Começo do documento
\begin{document}

	% Define algum espaçamento que eu não lembro, hehe :)
	\setlength\parskip{0.3cm}

	% Insere a Capa
	% Começo da folha de Capa
\begin{titlepage}

		% Título
		\title{
\textsc {\large Universidade de São Paulo\\
Instituto de Ciências Matemáticas e de Computação}\\[1cm]
{\large SCC0205 -- Teoria da Computação e Linguagens Formais}\\[5cm]
{\LARGE Trabalho 1 -- Analisador Léxico e Sintático para Linguagem AWK\\[4cm]}
		}

		% Autores
		\author{
Elias Italiano Rodrigues -- 7987251\\
Gabriel Tessaroli Giancristofaro -- 4321350\\
Paulo Augusto de Godoy Patire -- 7987060
		}

		% Inserção manual de data
		\date{
\vfill São Carlos, 2 de outubro de 2014
		}

		% Cria a Capa
		\maketitle
		\thispagestyle{empty}

% Fim da folha de Capa
\end{titlepage}

	
	% Reseta contador de página para 1 (assim não conta a Capa como página)
	\setcounter{page}{1}
	
	% Insere as outras partes do documento
	% Cria o Sumário
\tableofcontents

% Cria uma nova página, forçando o Sumário a ficar numa página separada
\newpage


	\section{Introdução \label{sec:introducao}}

Este trabalho implementa um analisador semântico com tratamento de erros para a linguagem de programação \texttt{LALG} utilizando as ferramentas \texttt{flex} e \texttt{bison}. Foram seguidas as instruções dadas em sala de aula assim como consultadas em manual~\cite{bib:manual} e em livro~\cite{bib:livro}.

	\section{Desenvolvimento \label{sec:desenvolvimento}}

\subsection{O simulador Amnesia \label{subsec:amnesia}}

De acordo com a página oficial do Amnesia \cite{bib:amnesia}:
\begin{quote}``O Amnesia é um simulador de hierarquia de memória, de sistemas computacionais, com fins didáticos. Ele permite simular o comportamento de registradores em um processador, memórias cache, memória principal e memória virtual paginada.

O Amnesia representa as estruturas de hardware e software usadas pela hierarquia de memória, a funcionalidade das mesmas e o impacto no desempenho quando esta hierarquia é usada. Diferentes configurações da hierarquia de memória podem ser estabelecidas e comparadas durante as atividades de simulação.''
\end{quote}

Isso faz desse simulador uma boa escolha para este trabalho, pois atende à necessidade de definir diferentes configurações de hierarquia de memória (arquiteturas) para as simulações. E ainda, é um \textit{software} desenvolvido e mantido pela própria instituição de ensino da disciplina deste trabalho:
\begin{quote}
``O simulador está em desenvolvimento, desde 2007, por alunos de graduação e da pós-graduação do Instituto de Ciências Matemáticas e de Computação (ICMC) da Universidade de São Paulo (USP), orientados pelos professores Paulo Sérgio Lopes de Souza e Sarita Mazzini Bruschi.''
\end{quote}

\textbf{Observação}: importante ressaltar que o Amnesia é um simulador com propósito didático que incorpora características de Objetos de Aprendizagem (OA) e Recursos Educacionais Abertos (REA) para auxiliar professores e alunos durante o processo de aprendizagem. Portanto, não é adequado usá-lo para um processo de \textit{benchmark} que também não é o propósito deste trabalho.

\subsection{Arquivos de \textit{trace} \label{subsec:trace}}

Arquivos de \textit{trace}, ou arquivos de rastro, são arquivos de texto ASCII que contém em cada linha uma dupla: rótulo (decimal) e endereço (hexadecimal). Qualquer outra informação é vista como um comentário. O rótulo representa uma operação de memória a ser feita sobre o endereço. O rótulo pode ser:
\begin{itemize}[noitemsep]
	\item 0: leitura de dados;
	\item 1: gravação de dados;
	\item 2: busca de instrução;
	\item 3: registro escape (tratado como tipo de acesso desconhecido);
	\item 4: registro escape (operação de cache flush).
\end{itemize}

\textbf{Observação}: como o objetivo deste trabalho é analisar a taxa de acerto e não foi considerada uma política de escrita para as \textit{caches}, então o rótulo utilizado para as operações é indiferente. Por definição, foi utilizado o \textbf{rótulo 2} para todas operações dos arquivos de \textit{trace}.

Os arquivos de \textit{trace} para este trabalho foram criados de tal modo que as heurísticas do princípio de localidade espacial e temporal fossem exploradas. Assim, criou-se quatro arquivos tais que: um não houvesse localidade espacial e nem temporal, um houvesse somente localidade espacial, um houvesse somente localidade temporal e outro com ambos os princípios de localidade. Além disso, criou-se um quinto arquivo de \textit{trace} com endereços aleatórios na tentativa de avaliar o algoritmo de substituição.

Para gerar os arquivos de \textit{trace}, usou-se um programa simples em linguagem C, \texttt{trace\char`_gen.c}, que imprime na saída padrão um arquivo de \textit{trace} com um dos casos citados acima.

Com exceção do primeiro \textit{trace}, a \textbf{quantidade de operações} foi definida em \textbf{32768}, pois assim é possível que um \textit{trace} acesse cada palavra da memória principal cujo tamanho está especificado mais adiante na Seção~\ref{subsec:arquiteturas}.

Devido ao tamanho dos arquivos de \textit{trace}, não é conveniente inseri-los inteiramente neste documento. Então seguem trechos dos arquivos com comentários sobre suas características.

\subsubsection{Não-espacial e não-temporal}

Gerado com:
\begin{verbatim}
    for (i = 0; i < 2978; i++) {
        printf("2 %x\n", (i * 11));
    }
\end{verbatim}

Trecho do arquivo \texttt{trace-00.txt}:
\begin{verbatim}
    2 0
    2 b
    2 16
    2 21
    2 2c
    2 37
    ...
\end{verbatim}

Neste \textit{trace} os endereços das palavras tem um intervalo suficientemente grande: 11 que é maior do que qualquer uma das quantidades de palavras por bloco presentes nas arquiteturas definidas. Isso proporciona característica não-espacial. Além disso, não é feito acesso repetido a um mesmo endereço o que proporciona característica não-temporal. Em particular, a quantidade de operações deste \textit{trace} é menor, pois caso contrário seriam gerados endereços de palavras inexistentes na memória principal.

\subsubsection{Espacial}

Gerado com:
\begin{verbatim}
    for (i = 0; i < 32768; i++) {
        printf("2 %x\n", i);
    }
\end{verbatim}

Trecho do arquivo \texttt{trace-01.txt}:

\begin{verbatim}
    2 0
    2 1
    2 2
    2 3
    2 4
    2 5
    2 6
    ...
\end{verbatim}

Neste \textit{trace} os endereços das palavras são bem próximos (consecutivos), porém com nenhum acesso repetido. Isso proporciona característica espacial e não-temporal.

\subsubsection{Temporal}

Gerado com:
\begin{verbatim}
    int *address = (int *)malloc(sizeof(int) * 96);

    address[0] = 0;
    for (i = 1; i < 96; i++) {
        address[i] = address[i-1] + 11;
    }
    for (i = 0; i < 32768; i++) {
        printf("2 %x\n", address[i % 96]);
    }

    free(address);
\end{verbatim}

Trecho do arquivo \texttt{trace-10.txt}:

\begin{verbatim}
    2 0
    2 b
    2 16
    2 21
    2 2c
    ...
    2 0  nonagésima sétima operação: começa a repetir
    2 b
    2 16
    ...
\end{verbatim}

Neste \textit{trace} os endereços são de palavras distantes (novamente 11 posições de distância que é suficiente para não serem do mesmo bloco em nenhuma das \textit{caches}), porém a cada 96 operações são realizados acessos repetidos aos mesmos endereços anteriores. Isso proporciona característica temporal e não-espacial.

\subsubsection{Espacial e temporal}

Gerado com:
\begin{verbatim}
    for (i = 0; i < 32768; i++) {
        printf("2 %x\n", i % 320);
    }
\end{verbatim}

Trecho do arquivo \texttt{trace-11.txt}:

\begin{verbatim}
    2 0
    2 1
    2 2
    2 3
    2 4
    2 5
    ...
    2 0 tricentésima vigésima primeira operação: começa a repetir
    2 1
    2 2
    ...
\end{verbatim}

Neste \textit{trace} os endereços das palavras são bem próximos (consecutivos) e ocorre repetição dos endereços a cada 320 operações. Isso proporciona característica espacial e temporal.

\subsubsection{Aleatório}

Gerado com:
\begin{verbatim}
    srand(time(NULL));
    for (i = 0; i < 32768; i++) {
        printf("2 %x\n", rand() % 192);
    }
\end{verbatim}

Trecho do arquivo \texttt{trace-rand.txt}:

\begin{verbatim}
    2 16
    2 2a
    2 59
    2 b9
    2 40
    2 65
    ...
\end{verbatim}

Neste \textit{trace} os endereços das palavras foram gerados aleatoriamente dentro um intervalo de 0 a 191 usando a função \texttt{rand()} da linguagem C.

\subsection{Arquiteturas \label{subsec:arquiteturas}}

Foram definidas uma arquitetura base e outras dez arquiteturas com configurações variadas a partir da arquitetura base. Toda \textbf{palavra} tem tamanho fixo de \textbf{4~bytes (32~bits)} e a \textbf{memória principal é endereçada a byte}.

Como citado anteriormente, o propósito deste trabalho e do simulador Amnesia não é o de \textit{benchmark}, portanto as configurações foram escolhidas de acordo com as limitações do simulador e também dos computadores pessoais do grupo do trabalho usados para executar o simulador. O tamanho da \textbf{memória principal} é fixo de \textbf{128KiB} (portanto 32768 palavras) e os tamanhos para as \textbf{\textit{caches}} podem ser de \textbf{512B}, \textbf{1KiB} e \textbf{2KiB}.

No Amnesia, o arquivo de arquitetura é definido no formato XML e um exemplo de arquitetura pode ser consultado no manual que acompanha o simulador.

\textbf{Observação}: as \textit{caches} definidas para este trabalho são todas do tipo \textbf{unified}, ou seja, não há separação dentro da \textit{cache} entre as palavras que representam instruções e as que representam dados como há no tipo \textit{split}. Como políticas de escrita não foram consideradas e por definição as operações de memória são apenas de busca de instrução (rótulo 2), não há motivo para escolher o tipo \textit{split}.

\subsubsection{Base \label{subsubsec:base}}

A arquitetura base foi definida no arquivo \texttt{arch-00.xml} e possui as configurações especificadas na Tabela~\ref{tab:base}.

\begin{table}[h!]
\centering
\caption{Configuração da arquitetura base.}
\label{tab:base}
\begin{tabular}{|c|l|c|c|c|}
\specialrule{2pt}{0pt}{0pt}
\multicolumn{5}{|c|}{\textbf{arch-00}}                                                                                                            \\ \hline
\textbf{Cache} & \multicolumn{1}{c|}{\textbf{Tamanho}} & \textbf{Palavras por bloco} & \textbf{Mapeamento} & \textbf{Substituição} \\ \hline
\textbf{L1} & 128 palavras  (512B)  & 2  & Direto & ----- \\ \hline
\end{tabular}
\end{table}


\subsubsection{Variadas \label{subsubsec:variadas}}

As dez arquiteturas variadas foram definidas nos respectivos arquivos \texttt{arch-01.xml}, \texttt{arch-02.xml}, \texttt{...}, \texttt{arch-10.xml} e possuem as configurações especificadas na Tabela~\ref{tab:variadas}.

\begin{table}[!htbp]
\centering
\caption{Configurações variadas. Em itálico estão os valores alterados com relação a arquitetura base.}
\label{tab:variadas}
\begin{tabular}{|c|l|c|c|c|}
\specialrule{2pt}{0pt}{0pt}
\multicolumn{5}{|c|}{\textbf{arch-01}} \\ \hline
\textbf{Cache} & \multicolumn{1}{c|}{\textbf{\textit{Tamanho}}} & \textbf{Palavras por bloco} & \textbf{Mapeamento} & \textbf{Substituição} \\ \hline
\textbf{L1} & \textit{256 palavras  (1KiB)}  & 2  & Direto & ----- \\ \hline
\multicolumn{1}{l}{} \\ \specialrule{2pt}{0pt}{0pt}
\multicolumn{5}{|c|}{\textbf{arch-02}} \\ \hline
\textbf{Cache} & \multicolumn{1}{c|}{\textbf{\textit{Tamanho}}} & \textbf{Palavras por bloco} & \textbf{Mapeamento} & \textbf{Substituição} \\ \hline
\textbf{L1} & \textit{512 palavras  (2KiB)}  & 2  & Direto & ----- \\ \hline
\multicolumn{1}{l}{} \\ \specialrule{2pt}{0pt}{0pt}
\multicolumn{5}{|c|}{\textbf{arch-03}} \\ \hline
\textbf{Cache} & \multicolumn{1}{c|}{\textbf{Tamanho}} & \textbf{\textit{Palavras por bloco}} & \textbf{Mapeamento} & \textbf{Substituição} \\ \hline
\textbf{L1} & 128 palavras  (512B)  & \textit{4}  & Direto & ----- \\ \hline
\multicolumn{1}{l}{} \\ \specialrule{2pt}{0pt}{0pt}
\multicolumn{5}{|c|}{\textbf{arch-04}} \\ \hline
\textbf{Cache} & \multicolumn{1}{c|}{\textbf{Tamanho}} & \textbf{\textit{Palavras por bloco}} & \textbf{Mapeamento} & \textbf{Substituição} \\ \hline
\textbf{L1} & 128 palavras  (512B)  & \textit{8}  & Direto & ----- \\ \hline
\multicolumn{1}{l}{} \\ \specialrule{2pt}{0pt}{0pt}
\multicolumn{5}{|c|}{\textbf{arch-05}} \\ \hline
\textbf{Cache} & \multicolumn{1}{c|}{\textbf{Tamanho}} & \textbf{Palavras por bloco} & \textbf{\textit{Mapeamento}} & \textbf{Substituição} \\ \hline
\textbf{L1} & 128 palavras  (512B)  & 2  & \textit{Associativo (2)} & FIFO \\ \hline
\multicolumn{1}{l}{} \\ \specialrule{2pt}{0pt}{0pt}
\multicolumn{5}{|c|}{\textbf{arch-06}} \\ \hline
\textbf{Cache} & \multicolumn{1}{c|}{\textbf{Tamanho}} & \textbf{Palavras por bloco} & \textbf{\textit{Mapeamento}} & \textbf{Substituição} \\ \hline
\textbf{L1} & 128 palavras  (512B)  & 2  & \textit{Associativo (4)} & FIFO \\ \hline
\multicolumn{1}{l}{} \\ \specialrule{2pt}{0pt}{0pt}
\multicolumn{5}{|c|}{\textbf{arch-07}} \\ \hline
\textbf{Cache} & \multicolumn{1}{c|}{\textbf{Tamanho}} & \textbf{Palavras por bloco} & \textbf{Mapeamento} & \textbf{\textit{Substituição}} \\ \hline
\textbf{L1} & 128 palavras  (512B)  & 2  & \textit{Associativo (4)} & \textit{LRU} \\ \hline
\multicolumn{1}{l}{} \\ \specialrule{2pt}{0pt}{0pt}
\multicolumn{5}{|c|}{\textbf{arch-08}} \\ \hline
\textbf{\textit{Cache}} & \multicolumn{1}{c|}{\textbf{Tamanho}} & \textbf{Palavras por bloco} & \textbf{Mapeamento} & \textbf{Substituição} \\ \hline
\textbf{L1} & 128 palavras (512B) & 2  & Direto & ----- \\ \hline
\textbf{\textit{L2}} & \textit{256 palavras (1KiB)} & 2  & Direto & ----- \\ \hline
\multicolumn{1}{l}{} \\ \specialrule{2pt}{0pt}{0pt}
\multicolumn{5}{|c|}{\textbf{arch-09}} \\ \hline
\textbf{\textit{Cache}} & \multicolumn{1}{c|}{\textbf{Tamanho}} & \textbf{Palavras por bloco} & \textbf{Mapeamento} & \textbf{Substituição} \\ \hline
\textbf{L1} & 128  palavras (512B) & 2  & Direto & ----- \\ \hline
\textbf{\textit{L2}} & \textit{256 palavras  (1KiB)} & 2  & Direto & ----- \\ \hline
\textbf{\textit{L3}} & \textit{512 palavras  (2KiB)} & 2  & Direto & ----- \\ \hline
\multicolumn{1}{l}{} \\ \specialrule{2pt}{0pt}{0pt}
\multicolumn{5}{|c|}{\textbf{arch-10}} \\ \hline
\textbf{\textit{Cache}} & \multicolumn{1}{c|}{\textbf{Tamanho}} & \textbf{Palavras por bloco} & \textbf{Mapeamento} & \textbf{Substituição} \\ \hline
\textbf{L1} & 128  palavras (512B) & 2  & Direto & ----- \\ \hline
\textbf{\textit{L2}} & \textit{256 palavras  (1KiB)} & \textit{4}  & \textit{Associativo (2)} & \textit{LRU} \\ \hline
\textbf{\textit{L3}} & \textit{512 palavras  (2KiB)} & \textit{8}  & \textit{Associativo (4)} & \textit{LRU} \\ \hline
\end{tabular}
\end{table}

\subsection{Resultados das execuções}

Para cada arquitetura, foram executados os cinco arquivos de \textit{trace} no simulador Amnesia. As taxas de acerto coletadas são mostradas na Tabela~\ref{tab:geral}.

\begin{table}[!htbp]
\centering
\caption{Taxa de acerto para cada arquitetura e arquivo de \textit{trace}.}
\label{tab:geral}
\begin{tabular}{|l|l|l|l|l|l|l|l|}
\hline
\multicolumn{1}{|c|}{\textbf{arch-}} & \textbf{Cache} & \multicolumn{1}{c|}{\textbf{trace-00}} & \multicolumn{1}{c|}{\textbf{trace-01}} & \multicolumn{1}{c|}{\textbf{trace-10}} & \multicolumn{1}{c|}{\textbf{trace-11}} & \multicolumn{1}{c|}{\textbf{trace-rand}} & \multicolumn{1}{c|}{\textbf{Média}} \\ \specialrule{2pt}{0pt}{0pt}
\textbf{00}                  & L1 & 0 & 0.5   & 0.332489    & 0.5        & 0.66537476 & 0.49946594 \\ \hline
\textbf{01}                  & L1 & 0 & 0.5   & 0.97628784  & 0.796875   & 0.9970703 & 0.817558285 \\ \hline
\textbf{02}                  & L1 & 0 & 0.5   & 0.97628784  & 0.9951172  & 0.9970703 & 0.867118835 \\ \hline
\textbf{03}                  & L1 & 0 & 0.75  & 0           & 0.75       & 0.66882324 & 0.54220581 \\ \hline
\textbf{04}                  & L1 & 0 & 0.875 & 0           & 0.875      & 0.6715698 & 0.60539245 \\ \hline
\textbf{05}                  & L1 & 0 & 0.5   & 0.020751953 & 0.5        & 0.6673279 & 0.42201996325 \\ \hline
\textbf{06}                  & L1 & 0 & 0.5   & 0           & 0.5        & 0.66674805 & 0.4166870125 \\ \hline
\textbf{07}                  & L1 & 0 & 0.5   & 0           & 0.5        & 0.66656494 & 0.416641235 \\ \specialrule{2pt}{0pt}{0pt}
\multirow{2}{*}{\textbf{08}} & L1 & 0 & 0.5   & 0.332489    & 0.5        & 0.66537476 & 0.49946594\\ \cline{2-8}
                             & L2 & 0 & 0     & 0.96447676  & 0.59375    & 0.99124485 & 0.6373679025\\ \specialrule{2pt}{0pt}{0pt}
\multirow{3}{*}{\textbf{09}} & L1 & 0 & 0.5   & 0.332489    & 0.5        & 0.66537476 & 0.49946594\\ \cline{2-8}
                             & L2 & 0 & 0     & 0.96447676  & 0.59375    & 0.99124485 & 0.6373679025\\ \cline{2-8}
                             & L3 & 0 & 0     & 0           & 0.97596157 & 0          & 0.2439903925\\ \specialrule{2pt}{0pt}{0pt}
\multirow{3}{*}{\textbf{10}} & L1 & 0 & 0.5   & 0.332489    & 0.5        & 0.66537476 & 0.49946594\\ \cline{2-8}
                             & L2 & 0 & 0.5   & 0.86983955  & 0.6982422  & 0.99562246 & 0.7659260525\\ \cline{2-8}
                             & L3 & 0 & 0.5   & 0.95539165  & 0.9919094  & 0.5        & 0.7368252625\\ \hline
\end{tabular}
\end{table}

\subsubsection{Relação da taxa de acerto com o tamanho da \textit{cache}}

Para avaliar o impacto do tamanho da \textit{cache} na taxa de acerto, contrastou-se as taxas de acerto das arquiteturas arch-00, arch-01 e arch-02. O resultado pode ser visto na Figura~\ref{fig:tamanho}.

Foi possível observar que o aumento no tamanho da \textit{cache} implicou no aumento da taxa de acerto como esperado. Quanto maior o tamanho da \textit{cache}, menor é a quantidade de falhas por capacidade, pois menos blocos disputam por uma mesma posição.

\begin{figure}[!htbp]
\centering
\begin{tikzpicture}
    \begin{axis}[
        width  = 0.8*\textwidth,
        height = 6cm,
        major x tick style = transparent,
        ybar=2*\pgflinewidth,
        bar width=14pt,
        ymajorgrids = true,
        ylabel = {Taxa de acerto},
        symbolic x coords={arch-$00$ ($512$B), arch-$01$ ($1$KiB), arch-$02$ ($2$KiB)},
        xtick = data,
        scaled y ticks = false,
        enlarge x limits=0.25,
        ymin=0,
        legend cell align=left,
        legend style={
                at={(1.25,0.15)},
                anchor=south east,
                column sep=1ex
        },
        legend entries={trace-$00$,trace-$01$,trace-$10$,trace-$11$, trace-rand,Média}
    ]
        \addplot[style={bblue,fill=bblue,mark=none}]
coordinates {(arch-$00$ ($512$B), 0)          (arch-$01$ ($1$KiB), 0)         (arch-$02$ ($2$KiB), 0)};
       
        \addplot[style={rred,fill=rred,mark=none}]
coordinates {(arch-$00$ ($512$B), 0.5)        (arch-$01$ ($1$KiB), 0.5)       (arch-$02$ ($2$KiB),0.5)};

        \addplot[style={ggreen,fill=ggreen,mark=none}]
coordinates {(arch-$00$ ($512$B), 0.332489)   (arch-$01$ ($1$KiB), 0.97628784) (arch-$02$ ($2$KiB),0.97628784)};

        \addplot[style={ppurple,fill=ppurple,mark=none}]
coordinates {(arch-$00$ ($512$B), 0.5)        (arch-$01$ ($1$KiB), 0.796875)   (arch-$02$ ($2$KiB),0.9951172)};

        \addplot[style={yyellow,fill=yyellow,mark=none}]
coordinates {(arch-$00$ ($512$B), 0.66537476) (arch-$01$ ($1$KiB), 0.9970703)  (arch-$02$ ($2$KiB),0.9970703)};

        \addplot[style={ggray,fill=ggray,mark=none}]
coordinates {(arch-$00$ ($512$B), 0.49946594) (arch-$01$ ($1$KiB), 0.817558285) (arch-$02$ ($2$KiB),0.867118835)};

    \end{axis}
\end{tikzpicture}
\caption{Comparação da taxa acerto com relação ao tamanho da \textit{cache}. \label{fig:tamanho}}
\end{figure}

\subsubsection{Relação da taxa de acerto com o tamanho do bloco}

Para avaliar o impacto do tamanho do bloco na taxa de acerto, contrastou-se as taxas de acerto das arquiteturas arch-00, arch-03 e arch-04. O resultado pode ser visto na Figura~\ref{fig:bloco}.

Quanto ao trace-rand, houve pouca variação na taxa de acerto, mas foi possível observar que a média de acertos aumentou conforme o tamanho do bloco.

Significativamente, observou-se aumento para os trace-01 e trace-11 e declínio para zero do trace-10. Isso aconteceu pois, uma vez que o tamanho do bloco foi aumentado, os trace-01 e trace-11 que possuem princípio de localidade espacial tem mais chance de acerto já que uma maior quantidade de palavras são copiadas para a \textit{cache}. Quanto ao declínio para zero do trace-10 deve-se ao fato dele possuir somente característica temporal, o que leva a um pior desempenho com blocos maiores, pois ocorre sobrescrita das palavras que possivelmente seriam utilizadas novamente durante a execução.

\begin{figure}[!htbp]
\centering
\begin{tikzpicture}
    \begin{axis}[
        width  = 0.8*\textwidth,
        height = 6cm,
        major x tick style = transparent,
        ybar=2*\pgflinewidth,
        bar width=14pt,
        ymajorgrids = true,
        ylabel = {Taxa de acerto},
        symbolic x coords={arch-$00$ ($2$ palavras), arch-$03$ ($4$ palavras), arch-$04$ ($8$ palavras)},
        xtick = data,
        scaled y ticks = false,
        enlarge x limits=0.25,
        ymin=0,
        legend cell align=left,
        legend style={
                at={(1.25,0.15)},
                anchor=south east,
                column sep=1ex
        },
        legend entries={trace-$00$,trace-$01$,trace-$10$,trace-$11$, trace-rand,Média}
    ]
        \addplot[style={bblue,fill=bblue,mark=none}]
coordinates {(arch-$00$ ($2$ palavras), 0)          (arch-$03$ ($4$ palavras), 0)         (arch-$04$ ($8$ palavras), 0)};
       
        \addplot[style={rred,fill=rred,mark=none}]
coordinates {(arch-$00$ ($2$ palavras), 0.5)        (arch-$03$ ($4$ palavras), 0.75)       (arch-$04$ ($8$ palavras),0.875)};

        \addplot[style={ggreen,fill=ggreen,mark=none}]
coordinates {(arch-$00$ ($2$ palavras), 0.332489)   (arch-$03$ ($4$ palavras), 0) (arch-$04$ ($8$ palavras),0)};

        \addplot[style={ppurple,fill=ppurple,mark=none}]
coordinates {(arch-$00$ ($2$ palavras), 0.5)        (arch-$03$ ($4$ palavras), 0.75)   (arch-$04$ ($8$ palavras),0.875)};

        \addplot[style={yyellow,fill=yyellow,mark=none}]
coordinates {(arch-$00$ ($2$ palavras), 0.66537476) (arch-$03$ ($4$ palavras), 0.66882324)  (arch-$04$ ($8$ palavras),0.6715698)};

        \addplot[style={ggray,fill=ggray,mark=none}]
coordinates {(arch-$00$ ($2$ palavras), 0.49946594) (arch-$03$ ($4$ palavras), 0.54220581) (arch-$04$ ($8$ palavras),0.60539245)};

    \end{axis}
\end{tikzpicture}
\caption{Comparação da taxa acerto com relação ao tamanho do bloco. \label{fig:bloco}}
\end{figure}

\subsubsection{Relação da taxa de acerto com o nível de associatividade}

Para avaliar o impacto do nível de associatividade na taxa de acerto, contrastou-se as taxas de acerto das arquiteturas arch-00, arch-05 e arch-06. O resultado pode ser visto na Figura~\ref{fig:associatividade}.

Quanto ao trace-01 e trace-11, que possuem característica espacial, não houve diferença na taxa de acerto, pois o tamanho do bloco não foi aumentado. Por outro lado, para o trace-10 que possui somente característica temporal, houve diminuição na taxa de acerto conforme o aumento do nível de associatividade. Esse comportamento é esperado para este \textit{trace}, pois foi aumentado o nível de associatividade sem aumentar o tamanho do bloco o que gerou mais falhas para localidade temporal.

\begin{figure}[!htbp]
\centering
\begin{tikzpicture}
    \begin{axis}[
        width  = 0.8*\textwidth,
        height = 6cm,
        major x tick style = transparent,
        ybar=2*\pgflinewidth,
        bar width=14pt,
        ymajorgrids = true,
        ylabel = {Taxa de acerto},
        symbolic x coords={arch-$00$ (direto), arch-$05$ (assoc. $2$), arch-$06$ (assoc. $4$)},
        xtick = data,
        scaled y ticks = false,
        enlarge x limits=0.25,
        ymin=0,
        legend cell align=left,
        legend style={
                at={(1.25,0.15)},
                anchor=south east,
                column sep=1ex
        },
        legend entries={trace-$00$,trace-$01$,trace-$10$,trace-$11$, trace-rand,Média}
    ]
        \addplot[style={bblue,fill=bblue,mark=none}]
coordinates {(arch-$00$ (direto), 0)          (arch-$05$ (assoc. $2$), 0)         (arch-$06$ (assoc. $4$), 0)};
       
        \addplot[style={rred,fill=rred,mark=none}]
coordinates {(arch-$00$ (direto), 0.5)        (arch-$05$ (assoc. $2$), 0.5)       (arch-$06$ (assoc. $4$),0.5)};

        \addplot[style={ggreen,fill=ggreen,mark=none}]
coordinates {(arch-$00$ (direto), 0.332489)   (arch-$05$ (assoc. $2$), 0.020751953) (arch-$06$ (assoc. $4$),0)};

        \addplot[style={ppurple,fill=ppurple,mark=none}]
coordinates {(arch-$00$ (direto), 0.5)        (arch-$05$ (assoc. $2$), 0.5)   (arch-$06$ (assoc. $4$),0.5)};

        \addplot[style={yyellow,fill=yyellow,mark=none}]
coordinates {(arch-$00$ (direto), 0.66537476) (arch-$05$ (assoc. $2$), 0.6673279)  (arch-$06$ (assoc. $4$),0.66674805)};

        \addplot[style={ggray,fill=ggray,mark=none}]
coordinates {(arch-$00$ (direto), 0.49946594) (arch-$05$ (assoc. $2$), 0.42201996325) (arch-$06$ (assoc. $4$),0.4166870125)};

    \end{axis}
\end{tikzpicture}
\caption{Comparação da taxa acerto com relação o nível de associatividade. \label{fig:associatividade}}
\end{figure}

\newpage
\subsubsection{Relação da taxa de acerto com o algoritmo de substituição}

Para avaliar a taxa de acerto com relação ao algoritmo de substituição, observou-se as arquiteturas arch-06 e arch-07. Pela Tabela~\ref{tab:geral}, pode-se notar que a diferença entre as duas foi mínima, havendo pequena alteração somente para o trace-rand. Isso mostra que os arquivos de \textit{trace} gerados não foram suficientes para avaliar o algoritmo de substituição. Com exceção do trace-rand, os demais \textit{traces} tem padrões de acesso sistemáticos o que implica na indiferença do algoritmo de substituição escolhido.

\subsubsection{Relação da taxa de acerto com o número de \textit{caches}}

Para avaliar o impacto do número de \textit{caches} na taxa de acerto, contrastou-se as taxas de acerto das arquiteturas arch-00, arch-08 e arch-09. Notoriamente o aumento do número de \textit{caches} diminui a quantidade de acessos à memória principal, pois proporciona mais alternativas de acesso. O problema que o número de \textit{caches} pode ocasionar é com relação à atualização dos dados (escrita) que não foi abordada nesse trabalho.

\subsubsection{Combinação das arquiteturas}

A arquitetura arch-10 proposta é uma combinação das demais com o objetivo de conseguir um melhor desempenho. A partir da Tabela~\ref{tab:geral} constatou-se que houve um melhor desempenho, pois com exceção das arquiteturas arch-01 e arch-02 que possuem tamanho da L1 maior, a arch-10 proporcionou, em média geral, uma taxa de acerto maior que todas as demais arquiteturas.
	\section{Resultados}

Os programas foram compilados e executados de modo automatizado por \emph{shell scripts} no computador de um dos integrantes do grupo do trabalho. Seguem as especificações desse computador:

\begin{verbatim}Phoronix Test Suite v5.2.1
System Information

Hardware:
Processor: Intel Core i7-3612QM @ 3.10GHz (8 Cores),
Motherboard: Dell 0DNMM8, Chipset: Intel 3rd Gen Core DRAM, Memory: 8192MB,
Graphics: Intel HD 4000 2048MB (1100MHz)

Software:
OS: Fedora 20, Kernel: 3.15.10-201.fc20.x86_64 (x86_64),
Compiler: GCC 4.8.3 20140624, File-System: ext4
\end{verbatim}

\subsection{Gauss-Legendre \emph{vs} Borwein (1984)}

A especificação do trabalho pede para que os programas retornem o número $\pi$ com precisão $d = 6$ casas corretas considerando um quantidade de iterações $N = 10^9$. Porém, para os algoritmos de Gauss-Legendre e Borwein (1984), essa precisão do $\pi$ é alcançada rapidamente com a apenas $N = 2$ iterações. Diante disso, com o objetivo de ``estressar" as implementações desses algoritmos, foram criados casos de testes para alcançar precisão $d = 10^i, i = 4, 5, 6, 7$.

Para conferir a corretude dos dígitos calculados, eles foram comparados com os dígitos gerados pelo programa \texttt{pi} da CLN\cite{programa-pi} da seguinte maneira: executou-se \texttt{pi} para as precisões $d = 10^i, i = 4, 5, 6, 7$ redirecionando a saída para arquivos nomeados de acordo com $d$, e então gerou-se uma lista \texttt{md5sum} desses arquivos à qual as saídas deste trabalho foram comparadas.

\begin{table}[h]
\begin{center}
	\begin{tabular}{ |c||c|c|c||c|c|c| } 
		\hline
		Precisão & GL & GLP &  \textit{Speedup} GL & B & BP & \textit{Speedup} B \\
		\hline

		$10^4$ & 0.05   & 0.03   & 1.66 & 0.09   & 0.09   & 1.00 \\
		$10^5$ & 0.89   & 0.82   & 1.09 & 2.26   & 2.38   & 0.95 \\
		$10^6$ & 21.53  & 20.79  & 1.04 & 54.36  & 48.58  & 1.12 \\
		$10^7$ & 365.77 & 330.42 & 1.1  & 938.84 & 765.62 & 1.23 \\
		
		\hline
	\end{tabular}
	\caption{Tempos (s) de execução e \emph{speed-up} dos algoritmos Gauss-Legendre e Borwein (1984)}
	\caption*{
		GL:  Gauss-Legendre Sequencial / GLP: Gauss-Legendre Paralelo\\
		B:   Borwein (1984) Sequencial / BP:  Borwein (1984) Paralelo
	}
\end{center}
\end{table}

\begin{figure}[h]
\centering
\begin{tikzpicture}
\begin{axis}[
	ylabel={Tempo para calcular (em segundos)},
	xlabel={Precisão do $\pi$ (em escala logarítmica base $10$)},
	legend cell align=left,
	legend pos=north west,
	ymajorgrids=true,
	xmajorgrids=true,
	grid style=dashed,
	/pgf/number format/1000 sep={},
	scale only axis,
	xtick = {4,5,6,7}
]
\addplot[color=red,mark=square]
coordinates{
	(4, 0.09)
	(5, 2.26)
	(6, 54.36)
	(7, 938.84)
};
\addlegendentry{Borwein (1984) -- Sequencial}

\addplot[color=blue,mark=square]
coordinates{
	(4, 0.09)
	(5, 2.38)
	(6, 48.58)
	(7, 765.62)
};
\addlegendentry{Borwein (1984) -- Paralelo}

\addplot[color=black,mark=square]
coordinates{
	(4, 0.05)
	(5, 0.89)
	(6, 21.53)
	(7, 365.77)
};
\addlegendentry{Guass-Legendre -- Sequencial}

\addplot[color=green,mark=square]
coordinates{
	(4, 0.03)
	(5, 0.82)
	(6, 20.79)
	(7, 330.42)
};
\addlegendentry{Guass-Legendre -- Paralelo}

\end{axis}
\end{tikzpicture}
\caption{Comparativo dos dados do cálculo do $\pi$ com Gauss-Legendre e Borwein (1984). \label{fig:gs-b}}
\end{figure}

Como podemos observar a partir de uma análise do \emph{speedup} de cada algoritmo, a implementação paralela torna-se mais rápida conforme a precisão requerida do $\pi$ aumenta. Porém, tal ganho do programa paralelo mostra-se ser pouco significativo se considerarmos a quantidade de \textit{threads} usadas -- 5 para Gauss-Legendre e 4 para Borwein (1984).

\subsection{Método de Monte Carlo}

Os testes do Monte Carlo para cálculo do $\pi$ foram feitos com uma quantidade fixa de $N = 10^9$ iterações, mas com uma quantidade váriavel de \textit{threads} o que nos possibilitou conferir o tempo de execução do programa de acordo com a quantidade de \textit{threads} utilizada. Pelos resultados da Tabela~\ref{tab:mc_res} pode-se perceber que o tempo de execução diminuiu conforme o aumento da quantidade de \textit{threads}, porém não de maneira proporcional.

\begin{table}[h]
\begin{center}
	\begin{tabular}{ |c|c|c|c|c| } 
		\hline
		nthreads & MC & MCP & \textit{Speedup} & $\approx \pi$ \\
		\hline

		$-$ & 29.68 & -      & -    & 3.16098686 \\
		$2$ & -     & 22.31  & 1.33 & 3.14160214 \\
		$4$ & -     & 20.61  & 1.44 & 3.14156532 \\
		$8$ & -     & 15.51  & 1.92 & 3.14165089 \\
		
		\hline
	\end{tabular}
	\caption{Tempos (s) de execução e \emph{speed-up} do Método de Monte Carlo para $\pi$ \label{tab:mc_res}}
	\caption*{
		MC:  Monte Carlo Sequencial\\
		MCP: Monte Carlo Paralelo
	}
\end{center}
\end{table}

\subsection{Black-Scholes}

Análogo aos testes do Método de Monte Carlo para $\pi$, os testes para Black-Scholes também foram feitos com uma quantidade variável de \textit{threads} e com uma entrada fixa do arquivo \texttt{entrada\char`_blackscholes.txt} que informa uma quantidade $M = 10^9$ iterações. Pelos resultados da Tabela~\ref{tab:bs_res} pode-se perceber que o tempo de execução diminuiu conforme o aumento da quantidade de \textit{threads}, porém também não de maneira proporcional.

\begin{table}[h]
\begin{center}
	\begin{tabular}{ |c|c|c|c|c| } 
		\hline
		nthreads & BS & BSP & \textit{Speedup} & intconf \\
		\hline

		$-$ & 115.95 & -     & -    & [99.986946,   99.986947] \\
		$2$ & -      & 74.67 & 1.55 & [99.989184,   99.989185] \\
		$4$ & -      & 59.66 & 1.94 & [99.983979,   99.983980] \\
		$8$ & -      & 47.62 & 2.43 & [100.000951, 100.000951] \\
		
		\hline
	\end{tabular}
	\caption{Tempos (s) de execução e \emph{speed-up} do Black-Scholes \label{tab:bs_res}}
	\caption*{
		BS:  Black-Scholes Sequencial\\
		BSP: Black-Scholes Paralelo\\
		intconf: intervalo de confiança
	}
\end{center}
\end{table}

	\section{Conclusão \label{sec:conclusao}}

O trabalho desenvolvido cumpre a especificação dada. Foi possível aprender mais sobre a ferramenta \texttt{flex} e concluir o analisador léxico de \texttt{LALG} que servirá como base para o próximo trabalho.

	\newpage

% Começo das Referências
\begin{thebibliography}{9}

	\bibitem{bib:open-mpi}
		Open MPI\\
		\textless\url{http://www.open-mpi.org/doc/v1.8/}\textgreater\\
		Acesso em: 6 de novembro de 2014.

	\bibitem{bib:cuda}
		Programming Guide :: CUDA Toolkit Documentation\\
		\textless\url{http://docs.nvidia.com/cuda/cuda-c-programming-guide/index.html}\textgreater\\
		Acesso em: 5 de dezembro de 2014.
	
	\bibitem{bib:livro-divisao}
		GRAMA, A. GUPTA, A. KARYPIS, G. and KUMAR, V.\\
		\textit{Introduction to Parallel Computing}. Pearson Education. (2rd ed.). p.101, p.141.

	\bibitem{bib:lena}
		The Lenna Story\\
		\textless\url{http://www.lenna.org/full/len_full.html}\textgreater\\
		Acesso em: 6 de novembro de 2014.

	\bibitem{bib:intro}
		An introduction to the Message Passing Interface (MPI) using C \\
		\textless\url{http://condor.cc.ku.edu/~grobe/docs/intro-MPI-C.shtml}\textgreater
		Acesso em: 9 de novembro de 2014.

% Fim das Referências
\end{thebibliography}

% http://upload.wikimedia.org/wikipedia/commons/8/8f/Whole_world_-_land_and_oceans_12000.jpg


% Fim do documento
\end{document}
