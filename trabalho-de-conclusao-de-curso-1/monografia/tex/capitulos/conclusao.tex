\chapter{Conclusão}
\label{ch:conclusao}

É inquestionável que para aumentar a escalabilidade da blockchain são necessárias mudanças no protocolo. Aprimoramentos tecnológicos somente não são suficientes para escalar o sistema. Por isso, modificações como o aumento do tamanho máximo do bloco são necessárias juntamente com outras inovações no protocolo.

Considerando os pensamentos da Escola Austríaca de economia, o livre mercado por enquanto é o arranjo que melhor define os preços de produtos e serviços e cria incentivos desfavoráveis à corrupção. Logo, um livre mercado é preferível a um planejamento central. O autor deste documento defende que este mesmo pensamento aplicado ao tamanho máximo do bloco resulta em melhores blocos e, quando somado às propostas da SegWit e Lightning Network, resulta em uma melhor escalabilidade para o Bitcoin.

O projeto de criptomoeda é intrinsecamente uma proposta de um livre mercado de moedas, pois permite a livre concorrência entre elas. E, por dentro de uma criptomoeda, devido ao seu mecanismo de consenso e ao seu código-fonte aberto, mudanças nos protocolos também estão suscetíveis a um livre mercado. Com isso, conclui-se que Bitcoin \textbf{é}  Unlimited. Isso ainda não é evidente, pois existe uma carência de profissionais capacitados para propor e implementar mudanças nos softwares (lembrando que Bitcoin é um software crítico). Conforme a oferta de profissionais capacitados na área aumentar, essa relação se tornará mais evidente.

No entanto, observa-se na comunidade do Bitcoin pessoas que, apesar de parecerem favoráveis ao livre mercado, propõem técnicas de intervencionismo para controlar o tamanho do bloco. Em um livre mercado de tamanhos máximos de blocos e taxas de transação, a tendência é que os valores convirjam para aqueles que melhor valorizam a criptomoeda e não prejudicam a rede. Outra preocupação é com o perigo de centralização da rede causado pela concentração do poder de \textit{hash} em poucos mineradores/\textit{pools}. Tal centralização não ocorreria em um ambiente livre, pois um minerador que busca pelo maior lucro limitaria seu poder de \textit{hash} para não tomar grande parte da rede. Caso o contrário, ele estaria contribuindo para seu próprio prejuízo, uma vez que, havendo notória centralização, a moeda seria desvalorada pelas demais entidades da rede --- desenvolvedores, usuários e \textit{stakeholders} --- que migrariam para uma criptomoeda concorrente.

Então, como passo gradual, pode-se adotar o Classic e, posteriormente, a migração para o Unlimited com alterações no protocolo para suportar SegWit e Lightning Network.

Ainda assim, toda hesitação quanto ao rumo do projeto Bitcoin é bastante compreensível, visto que más decisões podem causar grandes prejuízos às entidades envolvidas e, de modo geral, criptomoeda ainda é uma tecnologia nova e em constante debate e desenvolvimento. Ademais, não tão cedo altcoins conseguirão competir ao mesmo nível com o Bitcoin, uma vez que a maioria dos investimentos e dos desenvolvedores experientes estão focados nele. Espera-se por grandes mudanças nas próximas décadas e inovações no uso de blockchains como visionado pela Ethereum.

\section{Contribuições}

Este trabalho reúne e comenta sobre assuntos relacionados a área de Criptomoeda e serve para introduzir pessoas interessadas ao tema e ao problema atual de escalabilidade da blockchain por meio de algumas propostas. Além dos profissionais altamente experientes que investem em projetos de criptomoeda, espera-se que a pesquisa sobre criptomoedas no âmbito acadêmico, que aliás ainda é pequena no Brasil, cresça nos próximos anos e possa fortalecer a tecnologia com apoio da academia.

Pesquisar para este trabalho proporcionou bons momentos de estudos sobre tecnologias interessantes e com um evidente potencial disruptivo, podendo causar grandes mudanças sociais, assim como fez o surgimento da Internet.

\section{Trabalhos Futuros}

Aprofundado o conhecimento no protocolo e na arquitetura do Bitcoin, pode-se no futuro estagiar na área com desenvolvimento de aplicações ou aprofundar-se na pesquisa com simulações sobre as propostas de escalabilidade, ou estudos sobre a segurança da rede Bitcoin e criptografia ou ainda estudar o código-fonte e contribuir para o projeto.
