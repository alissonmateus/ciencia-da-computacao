% Inicializa o documento
% define papel, tamanho global de fonte, tipo de documento
\documentclass[a4paper, twoside, 12pt]{article}

	% Multicolunas
	\usepackage{multicol}

	% Pacotes usados
	\usepackage[utf8]{inputenc} % enconding de caracteres
	\usepackage[brazil]{babel}  % locale pt_BR
	\usepackage[lmargin=2cm, rmargin=2cm, tmargin=2cm, bmargin=2cm]{geometry} % margens da folha
	\usepackage{indentfirst} % sempre indenta o primeiro parágrafo

	% Matemática	
	\usepackage{amsmath}
	\usepackage{amsfonts}
	\usepackage{gensymb}
	
	% Tabela longa que quebra entre as paginas
	\usepackage{longtable}

	% Para links e url
%	\usepackage{hyperref}
	
	\usepackage{listings} % listagem de código-fonte
	\renewcommand*{\lstlistingname}{Listagem} % texto para listagem de código
	\usepackage{color} % cor para usar na listagem de código-fonte
%	\usepackage{svg}
	\usepackage{graphicx,xcolor} % para inserir imagens
	\usepackage[nottoc,notlot,notlof]{tocbibind} % adiciona o tópico Referências ao Sumário
	\usepackage{textcomp} % accesso \textquotesingle

	% Para desenhar grafos
	\usepackage{tikz}
	\usetikzlibrary{arrows,positioning,shapes,decorations}

	% Desenhar circulos
%	\usepackage{tkz-euclide}
%	\usetkzobj{all}

	% Escrever algoritmos em pseudo-código
%	\usepackage[portuguese,linesnumbered]{algorithm2e}

	% Tabelas
	\usepackage{booktabs}
	\usepackage{caption}

	% Estilos para usar nos grafos
	\tikzset{
		>=stealth',
		punkt/.style={
			rectangle,
			text centered,
			inner sep=0.7em,
			draw,
			fill=blue!5
		},
		pil/.style={
			->,
			thick,
			shorten <=2pt,
			shorten >=2pt
		}
	}

	% Para plotar gráficos
%	\usepackage{pgfplots}
%	\pgfplotsset{width=10cm,compat=1.9}

	% Define cores para o highlight de código-fonte
	\definecolor{dkgreen}{rgb}{0,0.6,0}
	\definecolor{gray}{rgb}{0.5,0.5,0.5}
	\definecolor{mauve}{rgb}{0.58,0,0.82}
	
	% Define configuração para listagem de código-fonte em linguagem C
	\lstset{
		frame=tb,
		language=C,
		aboveskip=2mm,
		belowskip=2mm,
		showstringspaces=false,
		columns=flexible,
		basicstyle={\small\ttfamily},
		numbers=none,
		keywordstyle=\color{blue},
		commentstyle=\color{dkgreen},
		stringstyle=\color{mauve},
		breaklines=true,
		breakatwhitespace=false,
		tabsize=4
	}

% Começo do documento
\begin{document}

	% Define algum espaçamento que eu não lembro, hehe :)
	\setlength\parskip{0.3cm}

	% Insere a Capa
	% Começo da folha de Capa
\begin{titlepage}

		% Título
		\title{
\textsc {\large Universidade de São Paulo\\
Instituto de Ciências Matemáticas e de Computação}\\[1cm]
{\large SCC0205 -- Teoria da Computação e Linguagens Formais}\\[5cm]
{\LARGE Trabalho 1 -- Analisador Léxico e Sintático para Linguagem AWK\\[4cm]}
		}

		% Autores
		\author{
Elias Italiano Rodrigues -- 7987251\\
Gabriel Tessaroli Giancristofaro -- 4321350\\
Paulo Augusto de Godoy Patire -- 7987060
		}

		% Inserção manual de data
		\date{
\vfill São Carlos, 2 de outubro de 2014
		}

		% Cria a Capa
		\maketitle
		\thispagestyle{empty}

% Fim da folha de Capa
\end{titlepage}

	
	% Reseta contador de página para 1 (assim não conta a Capa como página)
	\setcounter{page}{1}
	
	% Insere as outras partes do documento
	% Cria o Sumário
\tableofcontents

% Cria uma nova página, forçando o Sumário a ficar numa página separada
\newpage


	\section{Introdução \label{sec:introducao}}

Este trabalho implementa um analisador semântico com tratamento de erros para a linguagem de programação \texttt{LALG} utilizando as ferramentas \texttt{flex} e \texttt{bison}. Foram seguidas as instruções dadas em sala de aula assim como consultadas em manual~\cite{bib:manual} e em livro~\cite{bib:livro}.

	\section{Como Usar \label{sec:como-usar}}

\subsection{Compilação}

O trabalho entregue, como requisitado, já foi previamente compilado (\texttt{Linux}), não havendo necessidade de executar esse passo. Porém, caso queira ou precise compilar novamente, basta estar dentro do diretório do trabalho e executar:

	\indent\indent\texttt{make}

É necessário ter instalado o compilador \texttt{gcc}, a ferramenta \texttt{flex}, assim como o utilitário \texttt{make} em sistema operacional \texttt{Linux}.

\subsection{Execução}

Para executar o trabalho, basta estar dentro de seu diretório e executar:

	\indent\indent\texttt{./bin/main}

Dessa maneira, o programa \texttt{LALG} será lido da entrada padrão \texttt{stdin}.

Para executá-lo sobre um arquivo, basta redirecionar a entrada:

	\indent\indent\texttt{./bin/main < meu-programa.lalg}

No diretório \texttt{./test} encontram-se alguns exemplos de programa em \texttt{LALG} para testar.\\
Por exemplo:

	\indent\indent\texttt{./bin/main < ./test/programa1.lalg}

Opcionalmente, para rodar para todos os programas \texttt{.lalg} de \texttt{./test}, execute:

	\indent\indent\texttt{make run}

As saídas serão escritas em arquivos com sufixo \texttt{\char`_out} na própria pasta \texttt{./test}.

\newpage
\subsection{Exemplo de Execução}

Comando:
	
	\indent\indent\texttt{./bin/main < ./test/programa1.lalg}

Saída:
\begin{multicols}{2}
\begin{verbatim}
 2: program - program 
 2:   nome1 - IDENT (identificador) 
 2:       ; - SEMICOLON (ponto-virgula) 
 3:     var - var 
 3:       a - IDENT (identificador) 
 3:       , - COMMA (virgula) 
 3:      a1 - IDENT (identificador) 
 3:       , - COMMA (virgula) 
 3:       b - IDENT (identificador) 
 3:       : - COLON (dois-pontos) 
 3: integer - integer 
 3:       ; - SEMICOLON (ponto-virgula) 
 4:   begin - begin 
 5:    read - read 
 5:       ( - OPAR (abre parenteses) 
 5:       a - IDENT (identificador) 
 5:       ) - CPAR (fecha parenteses) 
 5:       ; - SEMICOLON (ponto-virgula) 
 6:      a1 - IDENT (identificador) 
 6:      := - ATR (atribuicao) 
 6:       a - IDENT (identificador) 
 6:       * - MULT (multiplicacao) 
 6:       2 - INTEGER (numero inteiro) 
 6:       ; - SEMICOLON (ponto-virgula) 
 8:   while - while 
 8:       ( - OPAR (abre parenteses) 
 8:      a1 - IDENT (identificador) 
 8:       > - GR (maior) 
 8:       0 - INTEGER (numero inteiro) 
 8:       ) - CPAR (fecha parenteses) 
 8:      do - do 
 9:   begin - begin 
10:   write - write 
10:       ( - OPAR (abre parenteses) 
10:      a1 - IDENT (identificador) 
10:       ) - CPAR (fecha parenteses) 
10:       ; - SEMICOLON (ponto-virgula) 
11:      a1 - IDENT (identificador) 
11:      := - ATR (atribuicao) 
11:      a1 - IDENT (identificador) 
11:       - - MINUS (subtracao) 
11:       1 - INTEGER (numero inteiro) 
11:       ; - SEMICOLON (ponto-virgula) 
12:     end - end 
12:       ; - SEMICOLON (ponto-virgula) 
14:     for - for 
14:       b - IDENT (identificador) 
14:      := - ATR (atribuicao) 
14:       1 - INTEGER (numero inteiro) 
14:      to - to 
14:      10 - INTEGER (numero inteiro) 
14:      do - do 
15:   begin - begin 
16:       b - IDENT (identificador) 
16:      := - ATR (atribuicao) 
16:       b - IDENT (identificador) 
16:       + - PLUS (adicao) 
16:       2 - INTEGER (numero inteiro) 
16:       ; - SEMICOLON (ponto-virgula) 
17:       a - IDENT (identificador) 
17:      := - ATR (atribuicao) 
17:       a - IDENT (identificador) 
17:       - - MINUS (subtracao) 
17:       1 - INTEGER (numero inteiro) 
17:       ; - SEMICOLON (ponto-virgula) 
18:     end - end 
18:       ; - SEMICOLON (ponto-virgula) 
20:      if - if 
20:       a - IDENT (identificador) 
20:      <> - DIFFERENT (diferente) 
20:       b - IDENT (identificador) 
20:    then - then 
21:   write - write 
21:       ( - OPAR (abre parenteses) 
21:       a - IDENT (identificador) 
21:       ) - CPAR (fecha parenteses) 
21:       ; - SEMICOLON (ponto-virgula) 
22:     end - end 
22:       . - DOT (ponto)
\end{verbatim}
\end{multicols}

Mais exemplos estão disponíves no diretório \texttt{./test}.

	\newpage
\section{Organização dos Arquivos \label{sec:organizacao-dos-arquivos}}

O diretório do trabalho está organizado da seguinte maneira:

\indent\indent\texttt{./bin} : diretório para arquivos binários executáveis.\\
\indent\indent\indent\texttt{|-- main} : o programa principal a ser executado para fazer a análise léxica.

\indent\indent\texttt{./doc} : diretório dos arquivos \LaTeX~fonte deste relatório.

\indent\indent\texttt{./src} : diretório com os códigos-fonte do programa.\\
\indent\indent\indent\texttt{|-- \texttt{lalg}.h} : definições dos \textit{tokens} da linguagem \texttt{LALG} e erros.\\
\indent\indent\indent\texttt{|-- \texttt{lalg}.l} : programa \texttt{Lex} para a linguagem \texttt{LALG}.\\
\indent\indent\indent\texttt{|-- \texttt{lalg}.c} : programa \texttt{C} gerado pelo \texttt{flex}.\\
\indent\indent\indent\texttt{|-- main.c} : programa principal que usa o \texttt{\texttt{lalg}.c}.

\indent\indent\texttt{./test} : diretório com exemplos de programa \texttt{LALG}.

\indent\indent\texttt{Makefile} : arquivo para automizar compilação e execução usando o utilitário \texttt{make}.

\indent\indent\texttt{\texttt{LALG}} : definição da linguagem \texttt{LALG}.

\indent\indent\texttt{RELATORIO.pdf} : este relatório PDF compilado a partir de \texttt{./doc}.

\indent\indent\texttt{README} : arquivo com instruções.

	\section{Decisões de Projeto \label{sec:decisoes-de-projeto}}

\subsection{\textit{Tokens}}

Para catalogar os \textit{tokens}, foi decido o formato \textless\texttt{token, simbolo\char`_token}\textgreater. Os símbolos foram definidos conforme mostra a Tabela~\ref{tab:tokens}.

\begin{table}[h]
\begin{center}
	\begin{tabular}{|c||l|l|} 
		\hline
		\textbf{\textit{Token}}      & \textbf{Símbolo}   & \textbf{Descrição}\\
		\hline
		\texttt{:}                   & \texttt{COLON}     & dois-pontos\\
		\texttt{;}                   & \texttt{SEMICOLON} & ponto-e-vírgula\\
		\texttt{.}                   & \texttt{DOT}       & ponto\\
		\texttt{,}                   & \texttt{COMMA}     & vírgula\\
		\texttt{(}                   & \texttt{OPAR}      & abre parênteses\\
		\texttt{)}                   & \texttt{CPAR}      & fecha parênteses\\
		\texttt{:=}                  & \texttt{ATR}       & atribuição\\
		\texttt{>=}                  & \texttt{GOE}       & maior ou igual\\
		\texttt{<=}                  & \texttt{LOE}       & menor ou igual\\
		\texttt{<>}                  & \texttt{DIFFERENT} & diferente\\
		\texttt{=}                   & \texttt{EQUAL}     & igual\\
		\texttt{>}                   & \texttt{GR}        & maior\\
		\texttt{<}                   & \texttt{LE}        & menor\\
		\texttt{+}                   & \texttt{PLUS}      & adição\\
		\texttt{-}                   & \texttt{MINUS}     & subtração\\
		\texttt{*}                   & \texttt{MULT}      & multiplicação\\
		\texttt{/}                   & \texttt{DIV}       & divisão\\
		\texttt{<ident>}             & \texttt{IDENT}     & identificador\\
		\texttt{<numero\char`_int>}  & \texttt{INTEGER}   & número inteiro\\
		\texttt{<numero\char`_real>} & \texttt{REAL}      & número real\\
		\texttt{<char>}              & \texttt{CHAR}      & caractere entre \texttt{\textquotesingle} escapado ou não com \texttt{\char`\\}\\
		\texttt{<reservado>}         & \texttt{RESERVED}  & qualquer palavra reservada\\
		\texttt{<desconhecido>}      & \texttt{UNKNOWN}   & um token desconhecido\\
		\hline
	\end{tabular}
	\caption{Descrição dos símbolos adotados para os \textit{tokens}. \indent Observação: os \textit{tokens} \texttt{<reservado>} e \texttt{<desconhecido>} não fazem parte da gramática \texttt{LALG} em si, mas foram definidos apenas para documentação, pois foram usados na implementação\label{tab:tokens}.}
\end{center}
\end{table}

\subsection{Palavras Reservadas}

Para catalogar as palavras reservadas, foi decido o formato\\ \textless\texttt{palavra\char`_reservada, palavra\char`_reservada}\textgreater. Segue a lista de palavras reservadas definidas:

\texttt{begin}, \texttt{char}, \texttt{const}, \texttt{do}, \texttt{else}, \texttt{end}, \texttt{for}, \texttt{function}, \texttt{if}, \texttt{integer}, \texttt{procedure}, \texttt{program}, \texttt{read}, \texttt{real}, \texttt{repeat}, \texttt{then}, \texttt{to}, \texttt{until}, \texttt{var}, \texttt{while}, \texttt{write}	

A tabela de palavras reservadas foi descrita diretamente no código-fonte (\textit{hard-coded}) para melhor desempenho e em ordem alfabética. Mais informações estão documentadas no arquivo \texttt{./src/main.c}.

Para conferir se um \textit{token} casado como \texttt{IDENT} é ou não uma palavra reservada, foi criada uma função que faz \textbf{busca binária} sobre a tabela de palavras reservadas. Devido à pequena quantidade de palavras reservadas, 21, esse método é satisfatório em sua eficiência que é $O(\log_221)$ no pior caso.

\subsection{Erros Léxicos}

Os códigos para os possíveis erros léxicos foram definidos como mostra a Tabela~\ref{tab:erros}. Também foram definidos comprimentos máximos para alguns \textit{tokens} listados na Tabela~\ref{tab:comprimentos}.

Usou-se \textbf{código} \texttt{C} para os erros referentes a comprimento e \textbf{expressões regulares} no próprio \texttt{Lex} \texttt{./src/lalg.l} para casar os erros referentes a má formação.

Para cada \textit{token} lido pela função que faz a análise léxica, ele é impresso na saída padrão \texttt{stdout} juntamente com informações descritivas dos erros ocorridos.

\begin{table}[h]
\begin{center}
	\begin{tabular}{|l||l|} 
		\hline
		\textbf{Código}                        & \textbf{Descrição}\\
		\hline
		\texttt{SUCCESS}                       &       nenhum erro\\
		\texttt{ERR\char`_BAD\char`_IDENT}     & idenfiticador mal formado\\
		\texttt{ERR\char`_MAX\char`_IDENT}     & idenfiticador muito grande\\
		\texttt{ERR\char`_BAD\char`_INTEGER}   & numero inteiro mal formado\\
		\texttt{ERR\char`_MAX\char`_INTEGER}   & numero inteiro muito grande\\
		\texttt{ERR\char`_BAD\char`_REAL}      & numero real mal formado\\
		\texttt{ERR\char`_MAX\char`_REAL}      & numero real muito grande\\
		\texttt{ERR\char`_BAD\char`_CHAR}      & caractere mal formado\\
		\texttt{ERR\char`_MAX\char`_CHAR}      & caractere muito grande\\
		\texttt{ERR\char`_CHAR\char`_EMPTY}    & caractere vazio\\
		\texttt{ERR\char`_CHAR\char`_BREAK}    & caractere nao inline\\
		\texttt{ERR\char`_CHAR\char`_OPEN}     & caractere nao fechado\\
		\texttt{ERR\char`_COMMENT\char`_BREAK} & comentario nao inline\\
		\texttt{ERR\char`_COMMENT\char`_OPEN}  & comentraio nao fechado\\
		\texttt{ERR\char`_UNKNOWN}             &    token desconhecido\\
		\hline
	\end{tabular}
	\caption{Descrição dos códigos de erro adotados\label{tab:erros}.}
\end{center}
\end{table}

\begin{table}[h]
\begin{center}
	\begin{tabular}{|l||c|l|} 
		\hline
		\textbf{Código}                     & \textbf{Valor}  & \textbf{\textit{Token}}\\
		\hline
	\texttt{MAX\char`_LENGTH\char`_IDENT}   & 13      & \texttt{<ident>}\\
	\texttt{MAX\char`_LENGTH\char`_INTEGER} & 13      & \texttt{<numero\char`_int>}\\
	\texttt{MAX\char`_LENGTH\char`_REAL}    & 13      & \texttt{<numero\char`_real>}\\
		\hline
	\end{tabular}
	\caption{Comprimentos máximos definidos para alguns \textit{tokens}\label{tab:comprimentos}.}
\end{center}
\end{table}

	\section{Observações}

Os seguintes problemas foram encontrados na gramática do LALG durante a implementação deste trabalho:

\begin{itemize}

	\item Uma função não possui corpo;
	
	\item O valor retornado de uma função não pode ser usado para ser atribuído a uma variável.

\end{itemize}

Essas faltas na gramática influenciarão na análise semântica.
	\section{Conclusão \label{sec:conclusao}}

O trabalho desenvolvido cumpre a especificação dada. Foi possível aprender mais sobre a ferramenta \texttt{flex} e concluir o analisador léxico de \texttt{LALG} que servirá como base para o próximo trabalho.

	\newpage

% Começo das Referências
\begin{thebibliography}{9}

	\bibitem{bib:open-mpi}
		Open MPI\\
		\textless\url{http://www.open-mpi.org/doc/v1.8/}\textgreater\\
		Acesso em: 6 de novembro de 2014.

	\bibitem{bib:cuda}
		Programming Guide :: CUDA Toolkit Documentation\\
		\textless\url{http://docs.nvidia.com/cuda/cuda-c-programming-guide/index.html}\textgreater\\
		Acesso em: 5 de dezembro de 2014.
	
	\bibitem{bib:livro-divisao}
		GRAMA, A. GUPTA, A. KARYPIS, G. and KUMAR, V.\\
		\textit{Introduction to Parallel Computing}. Pearson Education. (2rd ed.). p.101, p.141.

	\bibitem{bib:lena}
		The Lenna Story\\
		\textless\url{http://www.lenna.org/full/len_full.html}\textgreater\\
		Acesso em: 6 de novembro de 2014.

	\bibitem{bib:intro}
		An introduction to the Message Passing Interface (MPI) using C \\
		\textless\url{http://condor.cc.ku.edu/~grobe/docs/intro-MPI-C.shtml}\textgreater
		Acesso em: 9 de novembro de 2014.

% Fim das Referências
\end{thebibliography}

% http://upload.wikimedia.org/wikipedia/commons/8/8f/Whole_world_-_land_and_oceans_12000.jpg


% Fim do documento
\end{document}
