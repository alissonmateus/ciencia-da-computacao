% Inicializa o documento
% define papel, tamanho global de fonte, tipo de documento
\documentclass[a4paper, twoside, 12pt]{article}

	% Pacotes usados
	\usepackage[utf8]{inputenc} % enconding de caracteres
	\usepackage[brazil]{babel}  % locale pt_BR
	\usepackage[lmargin=2cm, rmargin=2cm, tmargin=2cm, bmargin=2cm]{geometry} % margens da folha
	\usepackage{indentfirst} % sempre indenta o primeiro parágrafo

	% Matemática	
	\usepackage{amsmath}
	\usepackage{amsfonts}
	\usepackage{gensymb}
	
	\usepackage{listings} % listagem de código-fonte
	\renewcommand*{\lstlistingname}{Listagem} % texto para listagem de código
	\usepackage{color} % cor para usar na listagem de código-fonte
	\usepackage{graphicx} % para inserir imagens
	\usepackage[nottoc,notlot,notlof]{tocbibind} % adiciona o tópico Referências ao Sumário
	\usepackage{textcomp} % accesso \textquotesingle

	% Para desenhar grafos
	\usepackage{tikz}
	\usetikzlibrary{arrows,positioning,shapes,decorations}

	% Desenhar circulos
	\usepackage{tkz-euclide}
	\usetkzobj{all}

	% Escrever algoritmos em pseudo-código
	\usepackage[portuguese,linesnumbered]{algorithm2e}

	% Tabelas
	\usepackage{booktabs}
	\usepackage{caption}
	
	% Sintaxe para BNF
	%\usepackage{syntax}

	% Estilos para usar nos grafos
	\tikzset{
		>=stealth',
		punkt/.style={
			rectangle,
			text centered,
			inner sep=0.7em,
			draw,
			fill=blue!5
		},
		pil/.style={
			->,
			thick,
			shorten <=2pt,
			shorten >=2pt
		}
	}

	% Para plotar gráficos
	\usepackage{pgfplots}
	\pgfplotsset{width=10cm,compat=1.9}

	% Define cores para o highlight de código-fonte
	\definecolor{dkgreen}{rgb}{0,0.6,0}
	\definecolor{gray}{rgb}{0.5,0.5,0.5}
	\definecolor{mauve}{rgb}{0.58,0,0.82}
	
	% Define configuração para listagem de código-fonte em linguagem C
	\lstset{
%		frame=tb,
		language=Awk,
		aboveskip=2mm,
		belowskip=2mm,
		showstringspaces=false,
%		columns=flexible,
		basicstyle={\small\ttfamily},
		numbers=none,
		keywordstyle=\color{blue},
		commentstyle=\color{dkgreen},
		stringstyle=\color{mauve},
		breaklines=true,
%		breakatwhitespace=false,
		tabsize=4
	}

% Começo do documento
\begin{document}

	% Define algum espaçamento que eu não lembro, hehe :)
	\setlength\parskip{0.3cm}

	% Insere a Capa
	% Começo da folha de Capa
\begin{titlepage}

		% Título
		\title{
\textsc {\large Universidade de São Paulo\\
Instituto de Ciências Matemáticas e de Computação}\\[1cm]
{\large SCC0205 -- Teoria da Computação e Linguagens Formais}\\[5cm]
{\LARGE Trabalho 1 -- Analisador Léxico e Sintático para Linguagem AWK\\[4cm]}
		}

		% Autores
		\author{
Elias Italiano Rodrigues -- 7987251\\
Gabriel Tessaroli Giancristofaro -- 4321350\\
Paulo Augusto de Godoy Patire -- 7987060
		}

		% Inserção manual de data
		\date{
\vfill São Carlos, 2 de outubro de 2014
		}

		% Cria a Capa
		\maketitle
		\thispagestyle{empty}

% Fim da folha de Capa
\end{titlepage}

	
	% Reseta contador de página para 1 (assim não conta a Capa como página)
	\setcounter{page}{1}
	
	% Insere as outras partes do documento
	% Cria o Sumário
\tableofcontents

% Cria uma nova página, forçando o Sumário a ficar numa página separada
\newpage


	\section{Parte 1}

\subsection{A Linguagem AWK}

AWK é uma linguagem de programação interpretada para processamento de texto
comumente usada para extração de dados em documentos estruturados em \emph{registro} e \emph{campo}.

Um registro é qualquer quantidade de informação que represente uma entidade
e um campo é uma parte constituinte dessa informação.
Por exemplo: um arquivo de texto onde cada linha contenha nomes completos de alunos (registro), e
os nomes e sobrenomes (campos) estão separados por espaço. Ou ainda, mais comum, um arquivo no
formato \emph{.cvs} (Comma-separated values).

Com AWK é possível manipular tais tipos de arquivos para gerar uma nova apresentação ou fazer
alterações sistemáticas nos dados.

\subsection{A Notação BNF}

A notação BNF foi usada para escrever a primeira gramática da linguagem.
Foram definidos os conjuntos dos terminais $V_t$ e dos não-terminais $V_n$,
assim como o conjunto de regras $P$ da gramática e o não-terminal inicial \textless program\textgreater.
Nessa notação foram usadas em geral as recursões à direita para definir as possíveis cadeias.

\subsection{Definição Formal da Gramática em Notação BNF}

\lstinputlisting[language={}]{../grammar/awk.bnf}

	\section{Parte 2}

Nesta parte do trabalho foram feitas as derivações de três programas em linguagem AWK 
usando a gramática BNF definida na Parte 1.

\subsection{Derivação do programa hello-world.awk}

Um programa que usa a pattern BEGIN para imprimir uma mensagem de ``Hello World!''

\lstinputlisting{../src/hello-world.awk}
\lstinputlisting[language={}]{../derivation/hello-world.txt}

\subsection{Derivação do programa linha-grande.awk}

Um programa que considera pequeno um registro que contenha menos de oito campos
e grande um registro que contenha oito ou mais campos.

\lstinputlisting{../src/linha-grande.awk}
\lstinputlisting[language={}]{../derivation/linha-grande.txt}

\newpage

\subsection{Derivação do programa inverte-nomes.awk}

Um programa simples que inverte o primeiro e o segundo campo
de cada registro de um arquivo.

\lstinputlisting{../src/inverte-nomes.awk}
\lstinputlisting[language={}]{../derivation/inverte-nomes.txt}


	\section{Parte 3}

\subsection{Notação EBNF}

Foi usada notação de Wirth em que \texttt{[ ]} representa opcionalidade, \texttt{\{ \}} represente zero ou mais repetições, \texttt{|} representa alternativa, \texttt{=} representa definição e \texttt{( )} agrupamento.

\subsection{Conversão para notação EBNF}

\lstinputlisting[language={}]{../grammar/awk.ebnf}

	\newpage

% Começo das Referências
\begin{thebibliography}{9}

	\bibitem{bib:open-mpi}
		Open MPI\\
		\textless\url{http://www.open-mpi.org/doc/v1.8/}\textgreater\\
		Acesso em: 6 de novembro de 2014.

	\bibitem{bib:cuda}
		Programming Guide :: CUDA Toolkit Documentation\\
		\textless\url{http://docs.nvidia.com/cuda/cuda-c-programming-guide/index.html}\textgreater\\
		Acesso em: 5 de dezembro de 2014.
	
	\bibitem{bib:livro-divisao}
		GRAMA, A. GUPTA, A. KARYPIS, G. and KUMAR, V.\\
		\textit{Introduction to Parallel Computing}. Pearson Education. (2rd ed.). p.101, p.141.

	\bibitem{bib:lena}
		The Lenna Story\\
		\textless\url{http://www.lenna.org/full/len_full.html}\textgreater\\
		Acesso em: 6 de novembro de 2014.

	\bibitem{bib:intro}
		An introduction to the Message Passing Interface (MPI) using C \\
		\textless\url{http://condor.cc.ku.edu/~grobe/docs/intro-MPI-C.shtml}\textgreater
		Acesso em: 9 de novembro de 2014.

% Fim das Referências
\end{thebibliography}

% http://upload.wikimedia.org/wikipedia/commons/8/8f/Whole_world_-_land_and_oceans_12000.jpg


% Fim do documento
\end{document}
