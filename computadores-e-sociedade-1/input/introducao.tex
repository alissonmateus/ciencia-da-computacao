\section{Introdução \label{sec:introducao}}

Este é um trabalho de pesquisa em que são apresentadas algumas informações a respeito de espionagem computacional por meio de \textit{sniffers}, com a apresentação de leis e de um estudo de caso envolvendo o tema.

\subsection{Espionagem de Atividade Computacional \label{sec:espionagem}}

O significado de espionagem consiste na prática de obter informações de caráter sigiloso relativas ao alvo sem sua prévia autorização. No caso, como se trata de espionagem de atividade computacional, esta é realizada sobre os dados armazenados em disco rígido ou sendo enviados ou recebidos através da rede.

A espionagem de atividade computacional é um termo fácil de ser definido, porém nem sempre é simples chegar a um consenso quando se trata de espionagem ou simplesmente de monitoramento. Seria espionagem apenas quando é feito um acesso externo à rede alvo? Poderia o monitoramento de um gerente de rede ser considerado espionagem?

\subsection{\textit{Sniffers} \label{sec:sniffers}}

Literalmente, \textit{sniffer} significa ``farejador", porém uma melhor tradução para o português seria ``escuta" ou ``grampo".

Sniffers podem ser implementados de duas maneiras: software ou hardware. Na primeira, um software é instalado diretamente no alvo ou em um outro sistema externo responsável por fazer o grampo. Já no caso de uma abordagem por hardware, um dispositivo físico é acoplado no meio onde a informação trafega. Em uma rede sem-fio, como os dados são transmitidos pelo ar, é mais simples capturar um pacote de dados, diferentemente de uma rede cabeada, onde o trafego é direcionado, normalmente é o cenário onde o uso do hardware de \textit{sniffer} é feito.~\cite{bib:kurose}
